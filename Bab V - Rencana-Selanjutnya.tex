% ==========================================
% BAB V RENCANA SELANJUTNYA
% ==========================================
\chapter{RENCANA SELANJUTNYA}
\label{chap:rencana-selanjutnya}
\section{Rencana Implementasi}
Rencana implementasi Tugas Akhir ini mengikuti tahapan \textit{Software Development Life Cycle (SDLC)}
yang telah dijelaskan sebelumnya (perencanaan, analisis, desain, implementasi, dan pengujian),
namun diterjemahkan menjadi rencana kerja terstruktur selama kurang lebih 12 minggu.
Berikut adalah detail rencana implementasi yang ditampilkan dalam bentuk tabel \textit{Gantt Chart}.

\scriptsize
\begin{longtable}{ | p{2cm} | *{20}{c|} }
    \caption{Rencana Implementasi Tahap Proposal \textit{Gantt Chart}}
    \label{tbl:gantt-long}                                                                                      \\

    \hline
    \textbf{Aktivitas}
     & \multicolumn{4}{c|}{\textbf{Sep 2025}}
     & \multicolumn{4}{c|}{\textbf{Okt 2025}}
     & \multicolumn{4}{c|}{\textbf{Nov 2025}}
     & \multicolumn{4}{c|}{\textbf{Des 2025}}
     & \multicolumn{4}{c|}{\textbf{Jan 2026}}                                                                   \\
    \hline

    \textbf{Minggu}
     & 1                                      & 2                   & 3                   & 4
     & 1                                      & 2                   & 3                   & 4
     & 1                                      & 2                   & 3                   & 4
     & 1                                      & 2                   & 3                   & 4
     & 1                                      & 2                   & 3                   & 4                   \\
    \hline
    \endfirsthead

    \caption[]{Rencana Implementasi (Gantt Chart) — lanjutan}                                                   \\

    \hline
    \textbf{Aktivitas}
     & \multicolumn{4}{c|}{\textbf{Sep 2025}}
     & \multicolumn{4}{c|}{\textbf{Okt 2025}}
     & \multicolumn{4}{c|}{\textbf{Nov 2025}}
     & \multicolumn{4}{c|}{\textbf{Des 2025}}
     & \multicolumn{4}{c|}{\textbf{Jan 2026}}                                                                   \\
    \hline

    \textbf{Minggu}
     & 1                                      & 2                   & 3                   & 4
     & 1                                      & 2                   & 3                   & 4
     & 1                                      & 2                   & 3                   & 4
     & 1                                      & 2                   & 3                   & 4
     & 1                                      & 2                   & 3                   & 4                   \\
    \hline
    \endhead

    \midrule
    \multicolumn{21}{r}{\textit{Bersambung ke halaman berikutnya}}                                              \\
    \endfoot

    \midrule
    \endlastfoot

    % ================== DATA TABEL ==================

    Penentuan Topik dan Dosen Pembimbing
     & \cellcolor{gray!50}
     & \cellcolor{gray!50}
     & \cellcolor{gray!50}
     & \cellcolor{gray!50}
     & \cellcolor{gray!50}
     & \cellcolor{gray!50}
     & \cellcolor{gray!50}
     & \cellcolor{gray!50}
     & \cellcolor{gray!50}
     &                                        &                     &
     &                                        &                     &                     &
     &                                        &                     &                     &                     \\ \hline

    Eksplorasi Topik
     &                                        & \cellcolor{gray!50}
     & \cellcolor{gray!50}
     &
     &                                        &                     &                     &
     &                                        &                     &                     &
     &                                        &                     &                     &
     &                                        &                     &                     &                     \\ \hline

    Kajian Literatur Awal
     &                                        &                     & \cellcolor{gray!50}
     & \cellcolor{gray!50}

     &                                        &                     &                     &
     &                                        &                     &                     &
     &                                        &                     &                     &
     &                                        &                     &                     &                     \\ \hline

    Latar Belakang
     &                                        &                     &                     & \cellcolor{gray!50}

     & \cellcolor{gray!50}
     &                                        &                     &
     &                                        &                     &                     &
     &                                        &                     &                     &
     &                                        &                     &                     &                     \\ \hline

    Studi Literatur
     &                                        &                     &                     &
     & \cellcolor{gray!50}
     & \cellcolor{gray!50}
     &                                        &
     &                                        &                     &                     &
     &                                        &                     &                     &
     &                                        &                     &                     &                     \\ \hline

    Pendahuluan
     &                                        &                     &                     &
     & \cellcolor{gray!50}
     & \cellcolor{gray!50}
     &                                        &
     &                                        &                     &                     &
     &                                        &                     &                     &
     &                                        &                     &                     &                     \\ \hline

    Analisis Masalah
     &                                        &                     &                     &
     &                                        & \cellcolor{gray!50}
     & \cellcolor{gray!50}
     & \cellcolor{gray!50}

     &                                        &                     &                     &
     &                                        &                     &                     &
     &                                        &                     &                     &                     \\ \hline

    Desain Solusi
     &                                        &                     &                     &
     &                                        &                     &                     &
     & \cellcolor{gray!50}
     & \cellcolor{gray!50}
     &                                        &
     &                                        &                     &                     &
     &                                        &                     &                     &                     \\ \hline

    Rencana Selanjutnya
     &                                        &                     &                     &
     &                                        &                     &                     &
     &                                        & \cellcolor{gray!50}
     & \cellcolor{gray!50}
     &
     &                                        &                     &                     &
     &                                        &                     &                     &                     \\ \hline

    Pengumpulan Proposal
     &                                        &                     &                     &
     &                                        &                     &                     &
     &                                        &                     &
     & \cellcolor{gray!50}

     & \cellcolor{gray!50}                    &                     &                     &
     &                                        &                     &                     &                     \\ \hline

    Seminar Proposal
     &                                        &                     &                     &
     &                                        &                     &                     &
     &                                        &                     &                     &
     &                                        & \cellcolor{gray!50}
     & \cellcolor{gray!50}
     &
     &                                        &                     &                     &                     \\ \hline

    Perancangan
    Arsitektur,
    Model Data,
    dan \textit{Flow Policy}
     &                                        &                     &                     &
     &                                        &                     &                     &
     &                                        &                     &                     &
     &                                        &                     &                     & \cellcolor{gray!50}
     & \cellcolor{gray!50}                    &                     &                     &                     \\ \hline

    Implementasi \textit{Backend}
     &                                        &                     &                     &
     &                                        &                     &                     &
     &                                        &                     &                     &
     &                                        &                     &                     &
     & \cellcolor{gray!50}                    & \cellcolor{gray!50} & \cellcolor{gray!50} & \cellcolor{gray!50} \\ \hline
\end{longtable}

\clearpage

\scriptsize
\begin{longtable}{ | p{2cm} | *{20}{c|} }
    \caption{Rencana Implementasi Tahap Pengembangan \textit{Gantt Chart}}
    \label{tbl:gantt-long-impl}                                                                                 \\

    \hline
    \textbf{Aktivitas}
     & \multicolumn{4}{c|}{\textbf{Feb 2025}}
     & \multicolumn{4}{c|}{\textbf{Mar 2025}}
     & \multicolumn{4}{c|}{\textbf{Apr 2025}}
     & \multicolumn{4}{c|}{\textbf{Mei 2025}}
     & \multicolumn{4}{c|}{\textbf{Jun 2025}}                                                                   \\ \hline

    \textbf{Minggu}
     & 1                                      & 2                   & 3                   & 4
     & 1                                      & 2                   & 3                   & 4
     & 1                                      & 2                   & 3                   & 4
     & 1                                      & 2                   & 3                   & 4
     & 1                                      & 2                   & 3                   & 4                   \\ \hline

    \endfirsthead

    \caption[]{Rencana Implementasi Tahap Pengembangan \textit{Gantt Chart} — lanjutan}                         \\

    \hline
    \textbf{Aktivitas}
     & \multicolumn{4}{c|}{\textbf{Feb 2025}}
     & \multicolumn{4}{c|}{\textbf{Mar 2025}}
     & \multicolumn{4}{c|}{\textbf{Apr 2025}}
     & \multicolumn{4}{c|}{\textbf{Mei 2025}}
     & \multicolumn{4}{c|}{\textbf{Jun 2025}}                                                                   \\ \hline

    \textbf{Minggu}
     & 1                                      & 2                   & 3                   & 4
     & 1                                      & 2                   & 3                   & 4
     & 1                                      & 2                   & 3                   & 4
     & 1                                      & 2                   & 3                   & 4
     & 1                                      & 2                   & 3                   & 4                   \\ \hline

    \endhead

    \midrule
    \multicolumn{21}{r}{\textit{Bersambung ke halaman berikutnya}}                                              \\
    \endfoot

    \midrule
    \endlastfoot

    % ============== DATA MULAI DI SINI ==============

    Implementasi Portal Pasien dan Klinik \textit{(Frontend)}
     & \cellcolor{gray!50}                    & \cellcolor{gray!50} & \cellcolor{gray!50} &
     &                                        &                     &                     &
     &                                        &                     &                     &
     &                                        &                     &                     &
     &                                        &                     &                     &                     \\ \hline

    Inisialisasi dan Integrasi dengan Server
     &                                        &                     &                     & \cellcolor{gray!50}
     & \cellcolor{gray!50}                    &                     &                     &
     &                                        &                     &                     &
     &                                        &                     &                     &
     &                                        &                     &                     &                     \\ \hline

    Implementasi \textit{Break-Glass} dan \textit{Immutable Audit Log}
     &                                        &                     &                     &
     &                                        & \cellcolor{gray!50} & \cellcolor{gray!50} &
     &                                        &                     &                     &
     &                                        &                     &                     &
     &                                        &                     &                     &                     \\ \hline

    \textit{Unit Testing} dan \textit{System Integration Testing}
     &                                        &                     &                     &
     &                                        &                     &                     & \cellcolor{gray!50}
     & \cellcolor{gray!50}                    &                     &                     &
     &                                        &                     &                     &
     &                                        &                     &                     &                     \\ \hline

    Penyempurnaan Sistem dan Dokumentasi Teknis
     &                                        &                     &                     &
     &                                        &                     &                     &
     &                                        & \cellcolor{gray!50} & \cellcolor{gray!50} &
     &                                        &                     &                     &
     &                                        &                     &                     &                     \\ \hline

    Penyusunan Laporan TA
     &                                        &                     &                     &
     &                                        &                     &                     &
     &                                        &                     &                     & \cellcolor{gray!50}
     & \cellcolor{gray!50}                    &                     &                     &
     &                                        &                     &                     &                     \\ \hline

    Sidang Tugas Akhir
     &                                        &                     &                     &
     &                                        &                     &                     &
     &                                        &                     &                     &
     &                                        & \cellcolor{gray!50} &                     &
     &                                        &                     &                     &                     \\ \hline
\end{longtable}

\normalsize

\clearpage
\section{Perangkat Keras yang Digunakan}
Implementasi sistem akan dilakukan menggunakan Apple MacBook Pro 16 Inch
dengan chip M1 Pro, RAM 16 GB, dan penyimpanan 1 TB sebagai mesin utama
pengembangan. Spesifikasi ini cukup untuk menjalankan \textit{environment}
pengembangan modern (\textit{Node.js}, \textit{Docker}, \textit{database}, \textit{code editor})
secara bersamaan tanpa \textit{bottleneck} berarti. Berikut adalah daftar spesifikasi perangkat keras yang digunakan dalam bentuk tabel.

\begin{table}[H]
    \caption{Perangkat Keras}
    \label{tbl:perangkat-keras}
    \begin{tabular}{ | p{3cm} | p{9.5cm} | }
        \hline
        \textbf{Komponen} & \textbf{Spesifikasi}                                    \\
        \hline

        Model             & Apple MacBook Pro 16-inch (M1 Pro)                      \\
        \hline

        Prosesor          &
        Apple M1 Pro, 10-core CPU (8 performance cores, 2 efficiency cores),
        16-core GPU, \textit{16-core Neural Engine}                                 \\
        \hline

        GPU               & 16-core GPU                                             \\
        \hline

        Neural Engine     & \textit{16-core Neural Engine}                          \\
        \hline

        Memori            & 16 GB \textit{unified memory}                           \\
        \hline

        Penyimpanan       & 1 TB SSD                                                \\
        \hline

        Layar             &
        16.2-inch Liquid Retina XDR, 3456 × 2234 pixel, P3 wide color, True Tone    \\
        \hline

        Sistem Operasi    & \textit{macOS} terbaru (misalnya \textit{macOS Sonoma}) \\
        \hline

        Konektivitas      & Wi-Fi 6, Bluetooth 5.0                                  \\
        \hline

        Port              &
        3× Thunderbolt 4 (USB-C), HDMI, MagSafe 3, headphone jack,
        SDXC card slot                                                              \\
        \hline
    \end{tabular}
\end{table}

\clearpage

\section{Perangkat Lunak dan Teknologi yang Digunakan}
Pengembangan sistem akan menggunakan tumpukan teknologi web modern yang
mendukung pengembangan cepat, modular, dan mudah diuji. Seluruh kode sumber
akan dikelola menggunakan \textit{Git} dengan repositori pada platform seperti
\textit{GitHub} atau \textit{GitLab}. \textit{Code editor} utama yang digunakan adalah
Visual Studio Code dengan ekstensi pendukung \textit{TypeScript} dan pengembangan \textit{backend}. Berikut adalah daftar perangkat lunak pendukung untuk pengembangan dengan teknologi yang digunakan dalam bentuk tabel.
\begin{table}[H]
    \caption{Perangkat Lunak dan Teknologi}
    \label{tbl:perangkat-lunak}
    \begin{tabular}{ | p{2.5cm} | p{4cm} | p{7.5cm} | }
        \hline
        \textbf{Kategori} & \textbf{Teknologi/Alat}                                                                                         & \textbf{Fungsi Utama} \\
        \hline

        Version Control
                          & \textit{Git} + \textit{GitHub}/\textit{GitLab}
                          & Manajemen versi kode, kolaborasi, dan \textit{backup} proyek                                                                            \\
        \hline

        Backend
                          & \textit{Node.js} (\textit{TypeScript}) + \textit{NestJS}
                          & Membangun \textit{Consent \& Policy Gateway} (\textit{API}, \textit{policy engine}, \textit{break-glass logic})                         \\
        \hline

        Frontend
                          & \textit{Next.js} (\textit{React}, \textit{TypeScript}) + \textit{Tailwind CSS}
                          & Membangun \textit{Portal Pasien} dan \textit{Portal Klinisi} berbasis web                                                               \\
        \hline

        Database
                          & \textit{PostgreSQL}
                          & Penyimpanan \textit{consent}, \textit{policy}, sesi \textit{break-glass}, dan \textit{audit log}                                        \\
        \hline

        API Testing
                          & \textit{Postman} / \textit{Insomnia} / \textit{Hoppscotch}
                          & Pengujian \textit{endpoint backend} dan verifikasi \textit{response}                                                                    \\
        \hline

        Container (opsional)
                          & \textit{Docker}
                          & Menjalankan \textit{FHIR server} atau \textit{database} dalam \textit{container} terisolasi                                             \\
        \hline

        Code Editor
                          & Visual Studio Code
                          & Lingkungan utama penulisan dan \textit{debugging} kode                                                                                  \\
        \hline
    \end{tabular}
\end{table}

\clearpage

\section{Rencana Evaluasi}
Rencana evaluasi sistem berfokus pada pengujian fungsional berdasarkan kebutuhan fungsional
(FR-1 hingga FR-6) serta verifikasi perilaku sistem pada berbagai skenario akses seperti akses
normal dengan \textit{consent} yang valid, akses tanpa \textit{consent}, dan akses menggunakan
mekanisme \textit{break-glass}. Pengujian dilakukan menggunakan pendekatan
\textit{black-box testing}, di mana fokus evaluasi berada pada masukan dan keluaran sistem tanpa
mempertimbangkan implementasi internal.

Selain itu, dilakukan pula pengamatan sederhana terhadap kebutuhan nonfungsional, seperti waktu
respon dan kelengkapan \textit{audit log}, untuk memastikan bahwa sistem memenuhi ekspektasi
minimal terkait performa dan \textit{auditability}. Evaluasi ini memberikan dasar objektif untuk
menilai apakah sistem telah berfungsi sesuai rancangan dan mampu mendukung kebutuhan klinis
serta privasi pasien secara efektif. Berikut adalah daftar pengujian untuk membuktikan kebutuhan fungsional terpenuhi dalam bentuk tabel.

\scriptsize
\begin{table}[H]
    \caption{Rencana Pengujian Fungsional}
    \label{tbl:pengujian-fungsional}
    \begin{tabular}{ | p{1cm} | p{2.5cm} | p{4.5cm} | p{4.5cm} | }
        \hline
        \textbf{ID}                                                                                       & \textbf{Deskripsi}                                            & \textbf{Langkah Uji} & \textbf{Hasil yang Diharapkan} \\
        \hline

        FR-1                                                                                              & Pengelolaan persetujuan pasien                                &
        1) Login sebagai pasien. 2) Membuat \textit{consent} baru. 3) Menampilkan ulang \textit{consent}. &
        \textit{Consent} tersimpan di basis data, muncul pada daftar, dan detail sesuai input.                                                                                                                                    \\
        \hline

        FR-2                                                                                              & Persetujuan granular berdasarkan jenis data dan \textit{role} &
        1) Pasien membuat \textit{consent} untuk dokter umum pada data laboratorium.
        2) Dokter umum mencoba akses.
        3) Psikiater mencoba akses yang sama.                                                             &
        Dokter umum diizinkan mengakses data laboratorium, sedangkan psikiater ditolak karena tidak sesuai \textit{consent}.                                                                                                      \\
        \hline

        FR-3                                                                                              & Penegakan \textit{policy} dan label DS4P                      &
        1) Menandai suatu \textit{FHIR Resource} dengan label DS4P.
        2) Klinisi dengan atribut tidak valid mencoba akses.
        3) Klinisi dengan atribut valid mencoba akses.                                                    &
        Akses pertama ditolak sesuai \textit{policy}, akses kedua diizinkan sesuai kombinasi \textit{consent + policy + DS4P}.                                                                                                    \\
        \hline

        FR-4                                                                                              & \textit{Break-Glass Access}                                   &
        1) Klinisi mencoba mengakses data sensitif tanpa \textit{consent}.
        2) Klinisi mengaktifkan \textit{Break-Glass}, mengisi alasan, dan melakukan OTP.
        3) Klinisi mengakses data selama masa TTL.                                                        &
        Klinisi menerima \textit{Break-Glass Token} dan dapat mengakses data sensitif dalam TTL, setelah TTL berakhir, akses ditolak.                                                                                             \\
        \hline

        FR-5                                                                                              & \textit{Immutable Audit Trail}                                &
        1) Melakukan operasi akses normal dan \textit{break-glass}.
        2) Memeriksa tabel \textit{audit log}.
        3) Menjalankan verifikasi \textit{hash-chain}.                                                    &
        Semua event tercatat lengkap dengan waktu, pengguna, dan tindakan; \textit{hash-chain} valid dan tidak menunjukkan manipulasi.                                                                                            \\
        \hline

        FR-6                                                                                              & Portal Pasien \& Portal Klinisi                               &
        1) Pasien login dan mengelola \textit{consent}.
        2) Klinisi login dan melakukan permintaan akses.
        3) Mengamati kemudahan navigasi.                                                                  &
        Kedua portal dapat digunakan tanpa error, seluruh fungsi utama berjalan dengan baik, dan antarmuka mudah dipahami.                                                                                                        \\
        \hline
    \end{tabular}
\end{table}
\normalsize
\clearpage

\section{Analisis Risiko}
Sebagai bagian dari upaya memastikan pelaksanaan Tugas Akhir berjalan secara
terukur dan terkendali, dilakukan proses \textit{Risk Assessment} untuk mengidentifikasi
risiko-risiko yang berpotensi menghambat penyelesaian proyek, baik dari sisi teknis,
manajemen waktu, maupun lingkungan pengembangan. Setiap risiko dinilai berdasarkan
besar dampak dan probabilitas terjadinya dengan skala 1--5, serta dirumuskan respons
mitigasinya secara ringkas untuk meminimalkan pengaruh negatif terhadap proses
implementasi. Berikut adalah daftar analisis risiko dalam bentuk tabel.

\scriptsize
\begin{table}[H]
    \caption{\textit{Risk Assessment}}
    \label{tbl:risk-assessment}
    \begin{tabular}{ | p{0.8cm} | p{3.5cm} | p{1.5cm} | p{2cm} | p{4cm} | }
        \hline
        \textbf{Kode}                                                                                     & \textbf{Deskripsi Risiko}    &
        \textbf{Dampak (1--5)}                                                                            & \textbf{Probabilitas (1--5)} &
        \textbf{Risk Response}                                                                                                             \\
        \hline

        R1                                                                                                &
        Kompleksitas integrasi \textit{FHIR} dan \textit{DS4P} yang berpotensi menyebabkan keterlambatan. &
        4                                                                                                 & 3                            &
        Fokus pada subset \textit{FHIR Resource}; lakukan \textit{spike} awal dan gunakan arsitektur modular.                              \\
        \hline

        R2                                                                                                &
        Implementasi \textit{policy engine} dan mekanisme \textit{break-glass} terlalu kompleks.          &
        5                                                                                                 & 3                            &
        Lakukan implementasi bertahap, batasi skenario darurat, dan gunakan aturan sederhana terlebih dahulu.                              \\
        \hline

        R3                                                                                                &
        Perubahan kebutuhan dari pembimbing yang dapat mengganggu jadwal implementasi.                    &
        5                                                                                                 & 2                            &
        Kunci FR/NFR lebih awal, catat seluruh perubahan melalui \textit{change log}, dan gunakan strategi \textit{Git branching}.         \\
        \hline

        R4                                                                                                &
        Risiko penggunaan data pasien nyata dalam pengujian.                                              &
        5                                                                                                 & 2                            &
        Gunakan data dummy sepenuhnya dan lakukan verifikasi ulang terhadap dataset sebelum pengujian.                                     \\
        \hline

        R6                                                                                                &
        Kerusakan \textit{development environment} (\textit{Node.js}, dependency, konfigurasi).           &
        3                                                                                                 & 2                            &
        \textit{Lock version}, gunakan \textit{Docker}, dan dokumentasikan proses \textit{setup} serta prosedur \textit{restore}.          \\
        \hline
    \end{tabular}
\end{table}
\normalsize
\textit{Risk Assessment} ini berfungsi sebagai panduan untuk menjaga pekerjaan tetap
berada dalam jalur yang realistis dan terkontrol. Dengan mengidentifikasi potensi
risiko sejak awal dan menyiapkan strategi mitigasi yang spesifik, proses
implementasi diharapkan dapat berjalan lebih stabil, minim hambatan, dan
menghasilkan prototipe \textit{Consent \& Policy Gateway} yang sesuai dengan tujuan
Tugas Akhir. Evaluasi risiko ini bersifat adaptif dan dapat diperbarui mengikuti
dinamika pekerjaan maupun masukan dari pembimbing.
