% ==========================================
% BAB IV DESAIN KONSEP SOLUSI
% ==========================================
\chapter{DESAIN KONSEP SOLUSI}
\label{chap:desain-konsep-solusi}
\section{Model Konseptual Sebelumnya}
Model konseptual sistem saat ini menggambarkan alur pertukaran data kesehatan
yang hanya memanfaatkan standar \textit{FHIR} sebagai mekanisme interoperabilitas
lintas fasilitas kesehatan. Pada kondisi ini, aplikasi klinik mengirimkan permintaan
ke sistem basis data terpusat untuk mengambil atau memperbarui data pasien, yang
kemudian diteruskan ke sistem rekam medis elektronik (RME) internal rumah sakit
dan penyimpanan data. Mekanisme kontrol akses yang digunakan masih bersifat
\textit{general consent}, sehingga pasien tidak memiliki kendali granular atas data mana
yang dapat diakses.

Di sisi privasi, perlindungan hanya dilakukan melalui \textit{masking} sederhana pada
tampilan nama pasien, misalnya “Irf** Mus*****,” tanpa segmentasi data sensitif
atau kebijakan keamanan berbasis metadata. Tidak terdapat mekanisme penegakan
kebijakan (\textit{policy enforcement}), pengelolaan persetujuan granular, maupun audit
mendalam sehingga sistem rentan terhadap akses yang tidak tepat.

\begin{figure}[h] % t = top, b = bottom, h = here
    \centering
    \captionsetup{justification=centering}
    \includegraphics[width=1\textwidth]{image/gambar-iv-1.png}
    \caption{Model konseptual sebelumnya \textit{FHIR} tanpa \textit{Consent Management System}}
    \label{gambar:model-konseptual-sebelumnya}
\end{figure}

\clearpage
\section{Model Konseptual Solusi}
Model konseptual solusi yang diusulkan menambahkan lapisan baru bernama
\textit{Consent \& Policy Gateway} sebagai pengendali akses sebelum permintaan data
mencapai server \textit{FHIR}. Lapisan ini terdiri dari beberapa komponen inti yang
meliputi \textit{Consent Management}, \textit{Policy Engine} berbasis \textit{ABAC},
\textit{DS4P Security Label Filter}, \textit{Break-Glass Handler}, dan
\textit{Immutable Audit Log}.

Pada lapisan antarmuka, disediakan \textit{Portal Pasien} untuk pengelolaan persetujuan
granular dan \textit{Portal Klinisi} untuk permintaan akses data. Seluruh permintaan, baik
normal maupun darurat, harus melalui \textit{gateway} ini untuk diverifikasi, diberikan
keputusan akses, dan dicatat untuk keperluan audit.

Dengan demikian, solusi ini meningkatkan privasi, akuntabilitas, dan kepatuhan
regulasi, sekaligus mempertahankan interoperabilitas \textit{FHIR}.

\begin{figure}[h] % t = top, b = bottom, h = here
    \centering
    \captionsetup{justification=centering}
    \includegraphics[width=1\textwidth]{image/gambar-iv-2.png}
    \caption{Model konseptual solusi \textit{FHIR} dengan \textit{Consent Management System}}
    \label{gambar:model-konseptual-solusi}
\end{figure}

\clearpage

\subsection{Solusi \textit{Consent Management System}}
Sistem \textit{Consent Management} dirancang untuk memberikan kontrol penuh kepada
pasien terhadap siapa yang dapat mengakses data medis mereka, jenis data apa yang
boleh diakses, serta untuk tujuan apa. Solusi ini memanfaatkan \textit{FHIR Consent
    Resource}, di mana setiap persetujuan dicatat secara terstruktur dan dapat diperbarui
secara dinamis oleh pasien.

\textit{Portal Pasien} memungkinkan pasien memberikan persetujuan granular berbasis
kategori data, misalnya rekam mental, data sensitif seksual, hasil laboratorium,
catatan obat, dan kategori lainnya kepada pengguna di pihak klinis seperti dokter
umum, psikiater, perawat, serta konteks penggunaan seperti perawatan aktif,
penelitian, atau konsultasi lintas fasilitas.

\textit{Consent \& Policy Gateway} kemudian akan mencocokkan permintaan akses dari
klinisi terhadap persetujuan yang tersimpan untuk menentukan apakah akses
diperbolehkan atau ditolak. Dengan demikian, sistem ini menggantikan
\textit{general consent} dengan \textit{fine-grained consent}, sekaligus mendukung prinsip
transparansi dan \textit{patient-centric care}.

\begin{figure}[h] % t = top, b = bottom, h = here
    \centering
    \captionsetup{justification=centering}
    \includegraphics[width=1\textwidth]{image/gambar-iv-3.png}
    \caption{Model Solusi \textit{Consent Management System}}
    \label{gambar:model-solusi-consent-management-system}
\end{figure}

\clearpage

\subsection{Solusi Protokol \textit{Break-Glass} (Akses Darurat) }
Solusi \textit{Break-Glass} dirancang untuk memberikan akses darurat yang cepat namun
tetap akuntabel terhadap data pasien. Tidak seperti sistem tradisional yang hanya
membuka seluruh akses, solusi ini menerapkan \textit{ABAC (Attribute-Based Access
    Control)} untuk mengatur akses berdasarkan atribut kontekstual seperti status
darurat, peran medis, hubungan dengan pasien, dan justifikasi klinis.

Ketika suatu permintaan akses ditolak karena tidak terdapat \textit{consent}, sistem akan
menampilkan opsi \textit{"Break-Glass"}. Klinisi harus memberikan alasan tertulis,
melakukan peningkatan autentikasi (misalnya \textit{OTP}), dan menerima \textit{Break-Glass
    Token} dengan batas waktu tertentu (\textit{TTL}, misalnya 15 menit). Token ini hanya
mengizinkan akses terhadap subset data tertentu dan tetap melalui pemeriksaan
\textit{DS4P} untuk mencegah akses ke kategori data yang sangat sensitif tanpa jalur
legal yang sesuai.

Seluruh aktivitas selama sesi darurat dicatat ke dalam \textit{immutable audit log} yang
membentuk rantai \textit{hash} untuk mendeteksi perubahan. Dengan demikian,
mekanisme ini menjaga keseimbangan antara keselamatan pasien dan perlindungan
privasi.

\begin{figure}[h] % t = top, b = bottom, h = here
    \centering
    \captionsetup{justification=centering}
    \includegraphics[width=1\textwidth]{image/gambar-iv-4.png}
    \caption{Model Solusi \textit{Break-Glass Protocol}}
    \label{gambar:model-solusi-break-glass-protocol}
\end{figure}
