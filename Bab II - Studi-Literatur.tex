% % ==========================================
% % BAB II STUDI LITERATUR
% % ==========================================
\chapter{STUDI LITERATUR}
\label{chap:studi-literatur}
\section{\textit{Break-Glass Protocol}}
\textit{Break-glass} adalah mekanisme atau protokol pemberian akses darurat terhadap rekam medis ketika pasien tidak dapat memberikan persetujuan eksplisit karena kondisi klinis yang mengancam jiwa. Pada situasi akut seperti stroke, ketersediaan data pasien secara cepat sangat krusial untuk proses triase, diagnosis, dan pemilihan pusat perawatan. Hal ini ditegaskan dalam penelitian \cite{deoliveira2020breakglass} bahwa akses cepat terhadap data medis pasien merupakan elemen kritis dalam proses penentuan prioritas perawatan, diagnosis awal, dan pemilihan fasilitas kesehatan yang tepat. Dalam kondisi darurat, sistem harus tetap memungkinkan tenaga medis mengakses informasi tersebut meskipun pasien tidak berada dalam kondisi untuk memberikan persetujuan secara langsung.

Permasalahan utama pada mekanisme \textit{break-glass} tradisional adalah sulitnya mencabut kembali akses setelah keadaan darurat selesai. Penelitian tersebut menyatakan bahwa sejumlah pendekatan \textit{break-glass} yang ada belum menyediakan mekanisme pencabutan hak akses yang efektif setelah situasi darurat berakhir, sehingga menimbulkan risiko akses berkepanjangan yang tidak lagi sesuai dengan kebutuhan klinis.

Untuk mengatasi hal tersebut, \cite{deoliveira2020breakglass} mengusulkan \textit{Red Alert Protocol (RAP)} berbasis \textit {Ciphertext-Policy Attribute-Based Encryption (CP-ABE)} dan \textit{temporary tokens}. Token darurat memungkinkan tenaga medis mengakses data terenkripsi hanya selama periode darurat dan dicabut secara otomatis setelah sesi perawatan berakhir.

\section{\textit{Blockchain}}
Blockchain adalah teknologi \textit{decentralized ledger} yang menyimpan catatan transaksi dalam bentuk blok yang saling terhubung secara kriptografis. Dalam konteks kesehatan, \textit{blockchain} menawarkan integritas dan ketertelusuran yang kuat untuk pengelolaan data yang sensitif.

\cite{udayakumar2019blockchain} mendefinisikan \textit{blockchain} sebagai buku besar digital yang tersebar dan tidak bergantung pada satu otoritas pusat. Catatan di dalamnya bersifat permanen dan tidak dapat diubah setelah divalidasi, serta menggunakan mekanisme kriptografi untuk menjamin proses autentikasi serta otorisasi.

Teknologi konvensional seperti server tunggal dan \textit{database} relasional memiliki risiko besar seperti kegagalan pusat \textit{(single point of failure)}, akses tidak sah, dan ketidakmampuan melacak perubahan. \cite{sarode2023blockchain} menunjukkan bahwa \textit{audit trail} sering kali disimpan dalam \textit{database} yang dapat dimodifikasi, sehingga keandalannya diragukan karena \textit{database} relasional rentan mengalami perubahan, baik oleh pihak internal maupun eksternal, sehingga mengurangi tingkat kepercayaan terhadap validitas catatan tersebut.

Dalam konteks kesehatan, \textit{blockchain} tidak digunakan untuk menyimpan seluruh rekam medis karena ukuran data yang besar dan pertimbangan privasi. Sebaliknya, yang disimpan biasanya adalah hash, pointer, atau ringkasan metadata yang memastikan integritas dan keaslian data pada sumber utamanya, misalnya server rumah sakit. Dengan cara ini, \textit{blockchain} berfungsi sebagai \textit{proof ledger} terdistribusi.

\section{\textit{Audit Trail}}
\textit{Audit trail} adalah catatan kronologis mengenai aktivitas dalam sistem, seperti akses, perubahan data, atau tindakan administratif. Dalam sistem rekam medis elektronik (EHR), \textit{audit trail} merupakan mekanisme utama untuk memastikan akuntabilitas, integritas, dan ketertelusuran. \cite{sarode2023blockchain} menjelaskan bahwa \textit{audit trail} adalah rangkaian catatan yang mendokumentasikan berbagai peristiwa dan perubahan yang terjadi dalam sistem, dan sebagian besar fasilitas layanan kesehatan diwajibkan untuk memelihara catatan tersebut bagi setiap rekam medis elektronik. Menurut \cite{sarode2023blockchain}, \textit{audit trail} yang bergantung pada satu server atau \textit{database} cenderung rentan, karena kegagalan pada satu titik tersebut dapat membuat catatan audit hilang atau dimodifikasi tanpa terdeteksi.

Oleh karena itu, integrasi \textit{audit trail} dengan \textit{blockchain} dianggap sebagai solusi modern. Dengan menyimpan \textit{hash} atau ringkasan aktivitas pada blockchain, \textit{audit trail} menjadi entitas jejak yang tidak bisa diubah, sehingga meningkatkan kepercayaan terhadap catatan tersebut. Selain itu waktu tercatat secara otomatis dan verifikasi dilakukan terdistribusi, memungkinkan ketahanan terhadap kegagalan sistem lokal. Dengan demikian, teori audit trail modern dalam domain kesehatan menekankan perpindahan dari penyimpanan terpusat menuju penyimpanan terdistribusi yang dapat diverifikasi oleh berbagai pihak.

\section{Penelitian Terkait}
\subsection{Interoperabilitas Data Kesehatan dan Standar FHIR}
Interoperabilitas merupakan kemampuan sistem berbeda untuk saling bertukar dan
menggunakan informasi dengan cara yang bermakna. Dalam bidang kesehatan,
interoperabilitas sangat penting untuk memastikan kesinambungan perawatan dan
efisiensi layanan medis. \textit{FHIR} (\textit{Fast Healthcare Interoperability Resources})
dikembangkan oleh \textit{Health Level Seven International (HL7)} sebagai standar terbaru
untuk pertukaran data kesehatan elektronik yang fleksibel dan berbasis web.
\textit{FHIR} dirancang dengan pendekatan sumber daya modular yang memungkinkan
representasi entitas medis seperti \textit{Patient}, \textit{Observation}, \textit{Condition}, dan \textit{Consent}
secara terpisah namun terhubung. Setiap sumber daya memiliki \textit{URL} unik dan
dapat diakses melalui \textit{RESTful API}, menggunakan format \textit{JSON} atau \textit{XML},
sehingga mendukung integrasi lintas platform dan perangkat \cite{ayaz2021fhir}.

Selain itu, penelitian Tabari dkk. \cite{tabari2024fhirreview} menunjukkan bahwa penerapan \textit{FHIR}
terbukti meningkatkan interoperabilitas semantik dan mempercepat pertukaran data
antar sistem medis yang heterogen. Mereka mengidentifikasi dua model
implementasi utama, yaitu model data \textit{static} dan \textit{dynamic}, serta menekankan bahwa
\textit{FHIR} membantu menghubungkan data dari berbagai sumber seperti rumah sakit,
laboratorium, dan sistem riset klinis. Namun, terdapat tantangan utama berupa
ketidakkonsistenan pemetaan data, keterbatasan interoperabilitas semantik lintas
sistem, dan kebutuhan pengelolaan privasi pasien yang lebih ketat. Hal ini menjadi
dasar bahwa standar \textit{FHIR} perlu dikombinasikan dengan kebijakan kontrol akses
dan segmentasi data yang tepat agar keamanan tetap terjaga.

\subsection{Segmentasi Data Sensitif dan \textit{Data Segmentation for Privacy (DS4P)}}
Dalam sistem kesehatan modern, tidak semua data pasien memiliki tingkat
sensitivitas yang sama. Informasi mengenai \textit{HIV}, kesehatan mental, dan catatan
reproduksi, misalnya, memerlukan perlakuan khusus dalam kontrol akses. Untuk
itu, \textit{HL7} mengembangkan konsep \textit{Data Segmentation for Privacy (DS4P)}, yaitu
mekanisme pelabelan keamanan (\textit{security labeling}) terhadap informasi atau bagian
data tertentu dalam \textit{FHIR} agar dapat dibatasi aksesnya sesuai peraturan dan
persetujuan pasien.

Panduan resmi \cite{hl7ds4p2025} menjelaskan bahwa setiap label keamanan di \textit{DS4P}
mengandung metadata tentang tingkat sensitivitas, kategori privasi, atau peraturan
yang mengikat suatu data. Label ini kemudian digunakan oleh sistem \textit{policy engine}
untuk menegakkan kebijakan akses. Dengan demikian, \textit{DS4P} tidak secara langsung
mengenkripsi data, tetapi menghubungkan sumber daya informasi dengan kerangka
kerja keamanan yang lebih luas melalui label semantik.

Hal ini relevan bagi penelitian ini karena sistem \textit{Consent \& Policy Gateway} yang
dirancang akan memanfaatkan prinsip \textit{DS4P} untuk mengatur siapa yang dapat
mengakses data sensitif, dengan menambahkan label keamanan dalam metadata
\textit{FHIR resource} seperti \textit{meta.security}.

\subsection{Model Kontrol Akses pada Rekam Medis Elektronik: RBAC hingga
    ABAC }
Kontrol akses adalah fondasi utama keamanan informasi kesehatan. Model
\textit{Role-Based Access Control (RBAC)} secara tradisional digunakan dalam sistem informasi
kesehatan karena kesederhanaannya. Hak akses diberikan berdasarkan peran
pengguna (misalnya dokter, perawat, atau staf admin). Namun, penelitian \cite{decarvalho2018rbac} menemukan bahwa \textit{RBAC} memiliki
keterbatasan signifikan untuk konteks layanan kesehatan modern yang bersifat
dinamis, seperti pengelolaan akses darurat, delegasi izin, dan interoperabilitas lintas
\textit{domain}.

Untuk mengatasi keterbatasan tersebut, model \textit{Attribute-Based Access Control (ABAC)}
dikembangkan dengan keputusan berbasis atribut subjek, objek, dan konteks lingkungan. Menurut \cite{deoliveira2023acabac}, \textit{ABAC} memungkinkan keputusan akses spesifik
yang mempertimbangkan situasi waktu nyata, seperti lokasi pengguna atau status
darurat pasien. Fleksibilitas inilah yang membuat \textit{ABAC} lebih cocok untuk sistem
rekam medis elektronik yang kompleks.

\subsection{Akses Darurat dan Model AC-ABAC}
Situasi gawat darurat menuntut sistem yang mampu memberikan akses cepat
terhadap data medis tanpa mengorbankan keamanan. Pendekatan konvensional
yang dikenal sebagai \textit{break-glass access} memberikan akses darurat tanpa
pembatasan granular, namun kerap disalahgunakan karena minimnya \textit{audit} dan
kontrol otomatis.

Model \textit{AC-ABAC (Acute-Care Attribute-Based Access Control)} yang
dikembangkan oleh \cite{deoliveira2023acabac} memperkenalkan mekanisme dinamis
di mana keputusan akses didasarkan pada atribut klinis. Akses darurat diberikan
secara sementara dan dicatat sepenuhnya melalui \textit{audit trail} yang tidak dapat
diubah, memastikan keseimbangan antara ketersediaan data dan privasi pasien.

\subsection{\textit{Consent Management} dan Peran Pasien}
Konsep \textit{Patient-Accessible Electronic Health Records (PAEHR)} menekankan hak
pasien untuk mengakses, memberi, atau mencabut izin atas data kesehatannya. \cite{kariotis2025mentalhealth} menunjukkan bahwa pemberian akses langsung kepada pasien
meningkatkan transparansi dan kepercayaan antara pasien dan penyedia layanan
kesehatan.

\textit{FHIR} menyediakan sarana teknis untuk merekam dan mengatur persetujuan pasien.
Namun, penerapan \textit{Consent} dalam banyak studi masih bersifat umum
(\textit{coarse-grained}) dan belum mendukung pengaturan granular oleh pasien sendiri.
Kondisi ini menjadi celah riset yang dijawab oleh penelitian ini melalui rancangan
\textit{Portal Pasien} yang memungkinkan kontrol persetujuan granular berbasis \textit{FHIR}.
