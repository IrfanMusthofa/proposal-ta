% % ==========================================
% % BAB II STUDI LITERATUR
% % ==========================================
\chapter{STUDI LITERATUR}
\label{chap:studi-literatur}
\section{\textit{Break-Glass Protocol}}
\textit{Break-glass} adalah mekanisme atau protokol pemberian akses darurat terhadap rekam medis ketika pasien tidak dapat memberikan persetujuan eksplisit karena kondisi klinis yang mengancam jiwa. Pada situasi akut seperti stroke, ketersediaan data pasien secara cepat sangat krusial untuk proses triase, diagnosis, dan pemilihan pusat perawatan. Hal ini ditegaskan dalam penelitian \cite{deoliveira2020breakglass} bahwa akses cepat terhadap data medis pasien merupakan elemen kritis dalam proses penentuan prioritas perawatan, diagnosis awal, dan pemilihan fasilitas kesehatan yang tepat. Dalam kondisi darurat, sistem harus tetap memungkinkan tenaga medis mengakses informasi tersebut meskipun pasien tidak berada dalam kondisi untuk memberikan persetujuan secara langsung.

Permasalahan utama pada mekanisme \textit{break-glass} tradisional adalah sulitnya mencabut kembali akses setelah keadaan darurat selesai. Penelitian tersebut menyatakan bahwa sejumlah pendekatan \textit{break-glass} yang ada belum menyediakan mekanisme pencabutan hak akses yang efektif setelah situasi darurat berakhir, sehingga menimbulkan risiko akses berkepanjangan yang tidak lagi sesuai dengan kebutuhan klinis.

Untuk mengatasi hal tersebut, \cite{deoliveira2020breakglass} mengusulkan \textit{Red Alert Protocol (RAP)} berbasis \textit {Ciphertext-Policy Attribute-Based Encryption (CP-ABE)} dan \textit{temporary tokens}. Token darurat memungkinkan tenaga medis mengakses data terenkripsi hanya selama periode darurat dan dicabut secara otomatis setelah sesi perawatan berakhir.

\clearpage

Berikut adalah diagram alur \textit{Red Alert Protocol (RAP)} yang diadaptasi dari \cite{deoliveira2020breakglass}:

\begin{figure}[h] % t = top, b = bottom, h = here
    \centering
    \captionsetup{justification=centering}
    \includegraphics[width=0.95\textwidth]{image/gambar-ii-1.png}
    \caption{Diagram \textit{Red Alert Protocol (RAP)} \cite{deoliveira2020breakglass}}
    \label{gambar:diagram-red-alert-protocol}
\end{figure}

Berdasarkan gambar di atas, protokol dimulai dengan fase \textit{Initialization} (a) di mana pasien ($u_i$) mengenkripsi \textit{Electronic Medical Records} (EMR) mereka dan menyimpannya di \textit{Cloud Service Provider} (CSP) menggunakan pesan $m_{store}$. Saat situasi kritis terjadi, alur berlanjut ke \textit{Emergency Session} (b). Pada tahap ini, pusat panggilan darurat ($s_e$) mengirimkan pesan \textit{break-glass} ($m_{break}$) kepada \textit{Master Authority} (MA) untuk memulai sesi darurat. Secara paralel, tim medis melakukan otentikasi lokasi melalui pertukaran pesan tantangan-respons ($m_{challenge}$ dan $m_{solution}$) untuk membuktikan kehadiran fisik mereka. Setelah validasi berhasil, MA mengirimkan kunci dekripsi dan token akses sementara kepada tim medis melalui pesan $m_{grant}$.

Setelah mendapatkan akses, protokol memasuki fase \textit{Process Data} (c). Tim medis menggunakan token yang valid untuk meminta ($m_{req}$) dan mengunduh ($m_{retrieve}$) data pasien terenkripsi dari CSP.

\clearpage

Data tersebut kemudian didekripsi secara lokal menggunakan kunci \textit{Ciphertext-Policy Attribute-Based Encryption} (CP-ABE) darurat, dan tim medis juga dapat menambahkan informasi klinis baru ke dalam sistem melalui pesan $m_{add}$. Protokol diakhiri dengan fase \textit{Leave Session} (d) ketika perawatan selesai atau pasien dipindahkan. Perwakilan tim ($s_h$) mengirimkan notifikasi status ($m_{info}$) ke MA, yang kemudian memicu pengiriman perintah pencabutan ($m_{revoke}$) ke CSP, sehingga token akses tim tersebut menjadi tidak valid secara instan.

\section{\textit{Blockchain}}
\textit{Blockchain} adalah teknologi \textit{decentralized ledger} yang menyimpan catatan transaksi dalam bentuk blok yang saling terhubung secara kriptografis. Dalam konteks kesehatan, \textit{blockchain} menawarkan integritas dan ketertelusuran yang kuat untuk pengelolaan data yang sensitif.

\cite{udayakumar2019blockchain} mendefinisikan \textit{blockchain} sebagai buku besar digital yang tersebar dan tidak bergantung pada satu otoritas pusat. Catatan di dalamnya bersifat permanen dan tidak dapat diubah setelah divalidasi, serta menggunakan mekanisme kriptografi untuk menjamin proses autentikasi serta otorisasi.

Teknologi konvensional seperti server tunggal dan \textit{database} relasional memiliki risiko besar seperti kegagalan pusat \textit{(single point of failure)}, akses tidak sah, dan ketidakmampuan melacak perubahan. \cite{sarode2023blockchain} menunjukkan bahwa \textit{audit trail} sering kali disimpan dalam \textit{database} yang dapat dimodifikasi, sehingga keandalannya diragukan karena \textit{database} relasional rentan mengalami perubahan, baik oleh pihak internal maupun eksternal, sehingga mengurangi tingkat kepercayaan terhadap validitas catatan tersebut.

Dalam konteks kesehatan, \textit{blockchain} tidak digunakan untuk menyimpan seluruh rekam medis karena ukuran data yang besar dan pertimbangan privasi. Sebaliknya, yang disimpan biasanya adalah hash, pointer, atau ringkasan metadata yang memastikan integritas dan keaslian data pada sumber utamanya, misalnya server rumah sakit. Dengan cara ini, \textit{blockchain} berfungsi sebagai \textit{proof ledger} terdistribusi.

\clearpage


\section{\textit{Audit Trail}}
\textit{Audit trail} adalah catatan kronologis mengenai aktivitas dalam sistem, seperti akses, perubahan data, atau tindakan administratif. Dalam sistem rekam medis elektronik (EHR), \textit{audit trail} merupakan mekanisme utama untuk memastikan akuntabilitas, integritas, dan ketertelusuran. \cite{sarode2023blockchain} menjelaskan bahwa \textit{audit trail} adalah rangkaian catatan yang mendokumentasikan berbagai peristiwa dan perubahan yang terjadi dalam sistem, dan sebagian besar fasilitas layanan kesehatan diwajibkan untuk memelihara catatan tersebut bagi setiap rekam medis elektronik. Menurut \cite{sarode2023blockchain}, \textit{audit trail} yang bergantung pada satu server atau \textit{database} cenderung rentan, karena kegagalan pada satu titik tersebut dapat membuat catatan audit hilang atau dimodifikasi tanpa terdeteksi.

Oleh karena itu, integrasi \textit{audit trail} dengan \textit{blockchain} dianggap sebagai solusi modern. Dengan menyimpan \textit{hash} atau ringkasan aktivitas pada blockchain, \textit{audit trail} menjadi entitas jejak yang tidak bisa diubah, sehingga meningkatkan kepercayaan terhadap catatan tersebut. Selain itu waktu tercatat secara otomatis dan verifikasi dilakukan terdistribusi, memungkinkan ketahanan terhadap kegagalan sistem lokal. Dengan demikian, teori audit trail modern dalam domain kesehatan menekankan perpindahan dari penyimpanan terpusat menuju penyimpanan terdistribusi yang dapat diverifikasi oleh berbagai pihak.

\clearpage

Berikut adalah diagram arsitektur \textit{blockchain} dalam sistem rekam medis elektronik yang diadaptasi dari \cite{sarode2023blockchain}:

\begin{figure}[h] % t = top, b = bottom, h = here
    \centering
    \captionsetup{justification=centering}
    \includegraphics[width=0.95\textwidth]{image/gambar-ii-2.png}
    \caption{Diagram arsitektur \textit{blockchain} dalam sistem rekam medis elektronik \cite{udayakumar2019blockchain}}
    \label{gambar:diagram-arsitektur-blockchain}
\end{figure}

Berdasarkan arsitektur sistem yang diusulkan pada Gambar \ref{gambar:diagram-arsitektur-blockchain}, alur dimulai dengan interaksi pasien ($Patient$) yang mengunjungi berbagai penyedia layanan kesehatan ($Hospital~1$, $Hospital~2$, $Hospital~3$). Seluruh pertukaran data antara entitas ini difasilitasi melalui antarmuka pengguna berbasis web (\textit{Web UI}) yang terintegrasi dengan basis data masing-masing rumah sakit. \textit{Web UI} ini berfungsi sebagai jembatan penghubung menuju lapisan logika bisnis yang dijalankan oleh \textit{Smart Contract}. \textit{Smart Contract} bertugas memvalidasi transaksi dan memastikan integritas data sebelum informasi tersebut dicatat secara permanen, tanpa memerlukan otoritas sentral atau pihak ketiga.

\clearpage

Pada lapisan infrastruktur data, sistem menggunakan teknologi \textit{Blockchain} untuk menyimpan jejak audit (\textit{audit trail}) dari setiap peristiwa medis. Informasi penting seperti detail kunjungan (\textit{Patient Visit Information}), data konsultasi (\textit{Consultation Information}), serta diagnosis dan resep (\textit{Diagnosis and Prescription}) direkam ke dalam blok-blok ($Block~1$ hingga $Block~N$) yang saling terhubung secara kriptografis. Karena sifat \textit{blockchain} yang \textit{immutable} (tidak dapat diubah) dan terdesentralisasi, mekanisme ini menjamin bahwa seluruh riwayat medis pasien tersimpan secara kronologis, transparan, dan aman dari manipulasi data.


\section{Penelitian Terkait}
\subsection{Interoperabilitas Data Kesehatan dan Standar FHIR}
Interoperabilitas merupakan kemampuan sistem berbeda untuk saling bertukar dan
menggunakan informasi dengan cara yang bermakna. Dalam bidang kesehatan,
interoperabilitas sangat penting untuk memastikan kesinambungan perawatan dan
efisiensi layanan medis. \textit{FHIR} (\textit{Fast Healthcare Interoperability Resources})
dikembangkan oleh \textit{Health Level Seven International (HL7)} sebagai standar terbaru
untuk pertukaran data kesehatan elektronik yang fleksibel dan berbasis web.
\textit{FHIR} dirancang dengan pendekatan sumber daya modular yang memungkinkan
representasi entitas medis seperti \textit{Patient}, \textit{Observation}, \textit{Condition}, dan \textit{Consent}
secara terpisah namun terhubung. Setiap sumber daya memiliki \textit{URL} unik dan
dapat diakses melalui \textit{RESTful API}, menggunakan format \textit{JSON} atau \textit{XML},
sehingga mendukung integrasi lintas platform dan perangkat \cite{ayaz2021fhir}.

Selain itu, tinjauan literatur sistematis oleh \cite{ayaz2021fhir} menegaskan bahwa FHIR memegang peranan vital dalam menyelesaikan masalah interoperabilitas data klinis di masa depan, terutama melalui dukungan terhadap teknologi pintar seperti \textit{mobile health apps} dan \textit{wearable devices}. Meskipun demikian, implementasi FHIR masih menghadapi tantangan signifikan, termasuk kompleksitas standar, hambatan adopsi di institusi kesehatan, serta kesulitan teknis dalam pemetaan dan migrasi data dari \textit{legacy systems} ke standar baru ini. Oleh karena itu, keberhasilan penerapan FHIR tidak hanya bergantung pada keunggulan teknisnya, tetapi juga memerlukan strategi penanganan yang tepat terkait keamanan data sensitif dan privasi dalam lingkungan berbasis \textit{cloud}.

\clearpage

Berikut adalah diagram arsitektur secara umum \textit{FHIR} yang diadaptasi dari \cite{ayaz2021fhir}:

\begin{figure}[h] % t = top, b = bottom, h = here
    \centering
    \captionsetup{justification=centering}
    \includegraphics[width=0.95\textwidth]{image/gambar-ii-3.png}
    \caption{Diagram arsitektur \textit{FHIR} secara umum \cite{ayaz2021fhir}}
    \label{gambar:diagram-arsitektur-fhir}
\end{figure}

Berdasarkan arsitektur umum yang diilustrasikan pada Gambar 1, standar \textit{Fast Health Interoperability Resources} (FHIR) dirancang di atas protokol web standar, khususnya menggunakan pendekatan \textit{Representational State Transfer} (REST) berbasis HTTP. Diagram tersebut menunjukkan bahwa setiap interaksi dimulai dengan \textit{HTTPRequest} yang mencakup URL dasar untuk identifikasi sumber daya unik, serta \textit{HTTPResponseHeader} yang membawa metadata penting seperti status, tipe konten (\textit{Content-Type}), lokasi, dan stempel waktu modifikasi terakhir (\textit{Last-Modified}).

Struktur permintaan dalam FHIR, yang digambarkan sebagai \textit{FHIRRequest}, memungkinkan pengolahan data secara \textit{batch} atau transaksi melalui mekanisme \textit{Bundle}. Dalam mekanisme ini, sebuah \textit{Bundle\_batch} dapat memuat beberapa entri permintaan sekaligus, di mana setiap entri memiliki metode HTTP dan URL spesifik yang mengarah pada \textit{Resource} kesehatan tertentu. Pendekatan modular berbasis sumber daya ini merupakan fitur pembeda utama FHIR dibandingkan standar HL7 versi sebelumnya.

Di sisi respons, \textit{FHIRResponse} dirancang untuk memberikan umpan balik yang terstruktur.Jika terjadi kesalahan atau sekadar memberikan informasi status operasi, sistem dapat mengembalikan \textit{OperationOutcome}. Untuk permintaan yang berhasil, sistem menghasilkan \textit{FHIRResponse\_batch} yang berisi \textit{ResponseBundle}, di mana setiap entri respons menyertakan \textit{Resource} yang diminta atau hasil dari operasi yang dilakukan. Hal ini menegaskan bahwa unit dasar transaksi dan pertukaran data dalam arsitektur ini adalah \textit{Resource} itu sendiri, yang berfungsi sebagai konsep diskrit terkecil dalam pertukaran data medis.

\subsection{Segmentasi Data Sensitif dan \textit{Data Segmentation for Privacy (DS4P)}}
Dalam sistem kesehatan modern, tidak semua data pasien memiliki tingkat
sensitivitas yang sama. Informasi mengenai \textit{HIV}, kesehatan mental, dan catatan
reproduksi, misalnya, memerlukan perlakuan khusus dalam kontrol akses. Untuk
itu, \textit{HL7} mengembangkan konsep \textit{Data Segmentation for Privacy (DS4P)}, yaitu
mekanisme pelabelan keamanan (\textit{security labeling}) terhadap informasi atau bagian
data tertentu dalam \textit{FHIR} agar dapat dibatasi aksesnya sesuai peraturan dan
persetujuan pasien.

Panduan resmi \cite{hl7ds4p2025} menjelaskan bahwa setiap label keamanan di \textit{DS4P}
mengandung metadata tentang tingkat sensitivitas, kategori privasi, atau peraturan
yang mengikat suatu data. Label ini kemudian digunakan oleh sistem \textit{policy engine}
untuk menegakkan kebijakan akses. Dengan demikian, \textit{DS4P} tidak secara langsung
mengenkripsi data, tetapi menghubungkan sumber daya informasi dengan kerangka
kerja keamanan yang lebih luas melalui label semantik.

Hal ini relevan bagi penelitian ini karena sistem \textit{Consent \& Policy Gateway} yang
dirancang akan memanfaatkan prinsip \textit{DS4P} untuk mengatur siapa yang dapat
mengakses data sensitif, dengan menambahkan label keamanan dalam metadata
\textit{FHIR resource} seperti \textit{meta.security}.


\clearpage

\subsection{Model Kontrol Akses pada Rekam Medis Elektronik: RBAC hingga
    ABAC }
Kontrol akses adalah fondasi utama keamanan informasi kesehatan. Model
\textit{Role-Based Access Control (RBAC)} secara tradisional digunakan dalam sistem informasi
kesehatan karena kesederhanaannya. Hak akses diberikan berdasarkan peran
pengguna (misalnya dokter, perawat, atau staf admin). Namun, penelitian \cite{decarvalho2018rbac} menemukan bahwa \textit{RBAC} memiliki
keterbatasan signifikan untuk konteks layanan kesehatan modern yang bersifat
dinamis, seperti pengelolaan akses darurat, delegasi izin, dan interoperabilitas lintas
\textit{domain}.

Untuk mengatasi keterbatasan tersebut, model \textit{Attribute-Based Access Control (ABAC)} dikembangkan dengan keputusan berbasis atribut subjek, objek, dan konteks lingkungan. Menurut \cite{deoliveira2023acabac}, \textit{ABAC} memungkinkan keputusan akses spesifik
yang mempertimbangkan situasi waktu nyata, seperti lokasi pengguna atau status
darurat pasien. Fleksibilitas inilah yang membuat \textit{ABAC} lebih cocok untuk sistem
rekam medis elektronik yang kompleks.

Berikut adalah diagram alur kerja informasi perawatan akut pada \textit{Attribute-Based Access Control (ABAC)} yang diadaptasi dari \cite{deoliveira2023acabac}:

\begin{figure}[h] % t = top, b = bottom, h = here
    \centering
    \captionsetup{justification=centering}
    \includegraphics[width=1\textwidth]{image/gambar-ii-4.png}
    \caption{Diagram alur kerja informasi perawatan akut pada \textit{Attribute-Based Access Control (ABAC)} \cite{deoliveira2023acabac}}
    \label{gambar:diagram-alur-kerja-abac}
\end{figure}

Berdasarkan diagram alur kerja informasi perawatan akut, proses dimulai ketika \textit{Call Centre Team} menginisiasi \textit{Emergency Session (ES)}, yang secara otomatis memicu mesin ABAC untuk memberikan hak akses \textit{READ} dan \textit{UPDATE} kepada mereka. Tim ini kemudian mengundang \textit{Ambulance Team} ke dalam sesi. Setelah undangan diterima, \textit{Ambulance Team} awalnya hanya diberikan hak akses \textit{READ}, namun hak ini ditingkatkan menjadi \textit{READ} dan \textit{UPDATE} segera setelah mereka melakukan notifikasi penjemputan pasien \textit{pick up patient}. Mekanisme ini memastikan bahwa hak akses diberikan secara dinamis dan bertahap sesuai dengan keterlibatan aktif tim dalam menangani pasien.

Selanjutnya, alur berlanjut ketika \textit{Ambulance Team} mengundang \textit{Hospital Team}. Tim rumah sakit mendapatkan hak akses \textit{READ} saat menerima undangan untuk persiapan, dan kemudian mendapatkan akses penuh (\textit{READ} dan \textit{UPDATE}) saat pasien tiba dan diterima secara fisik di rumah sakit. Alur kerja ini diakhiri dengan fase pemulangan pasien (\textit{discharge}), di mana sistem memberikan waktu tambahan (\textit{extra time}) bagi tim medis untuk melengkapi data administratif atau klinis sebelum akses benar-benar dicabut dan sesi diakhiri (\textit{End ES}) secara permanen.

\subsection{Akses Darurat dan Model AC-ABAC}
Situasi gawat darurat menuntut sistem yang mampu memberikan akses cepat
terhadap data medis tanpa mengorbankan keamanan. Pendekatan konvensional
yang dikenal sebagai \textit{break-glass access} memberikan akses darurat tanpa
pembatasan granular, namun kerap disalahgunakan karena minimnya \textit{audit} dan
kontrol otomatis.

Model \textit{AC-ABAC (Acute-Care Attribute-Based Access Control)} yang
dikembangkan oleh \cite{deoliveira2023acabac} memperkenalkan mekanisme dinamis
di mana keputusan akses didasarkan pada atribut klinis. Akses darurat diberikan
secara sementara dan dicatat sepenuhnya melalui \textit{audit trail} yang tidak dapat
diubah, memastikan keseimbangan antara ketersediaan data dan privasi pasien.

\subsection{\textit{Consent Management} dan Peran Pasien}
Konsep \textit{Patient-Accessible Electronic Health Records (PAEHR)} menekankan hak
pasien untuk mengakses, memberi, atau mencabut izin atas data kesehatannya. \cite{kariotis2025mentalhealth} menunjukkan bahwa pemberian akses langsung kepada pasien
meningkatkan transparansi dan kepercayaan antara pasien dan penyedia layanan
kesehatan.

\textit{FHIR} menyediakan sarana teknis untuk merekam dan mengatur persetujuan pasien.
Namun, penerapan \textit{Consent} dalam banyak studi masih bersifat umum
(\textit{coarse-grained}) dan belum mendukung pengaturan granular oleh pasien sendiri.
Kondisi ini menjadi celah riset yang dijawab oleh penelitian ini melalui rancangan
\textit{Portal Pasien} yang memungkinkan kontrol persetujuan granular berbasis \textit{FHIR}.
