% ==========================================
% BAB I PENDAHULUAN
% ==========================================
\chapter{PENDAHULUAN}
\label{chap:pendahuluan}
% --- Latar Belakang ---
\section{Latar Belakang}
Perkembangan sistem informasi kesehatan di Indonesia mencapai tonggak penting dengan hadirnya SATUSEHAT, platform nasional berbasis standar
\textit{Fast Healthcare Interoperability Resources} (FHIR) yang bertujuan mewujudkan interoperabilitas rekam medis elektronik (RME) lintas fasilitas kesehatan. Dengan FHIR, data pasien direpresentasikan dalam bentuk sumber daya seperti \textit{Patient}, \textit{Observation}, dan \textit{Condition}, sehingga memungkinkan pertukaran data medis secara terstandar dan aman antar sistem yang heterogen \cite{ayaz2021fhir}. Standar ini menggabungkan fleksibilitas teknologi web modern dengan model data klinis granular, menjadikannya fondasi utama interoperabilitas semantik pada berbagai sistem kesehatan global \cite{tabari2024fhirreview}.

Meskipun demikian, interoperabilitas teknis saja belum cukup tanpa tata kelola akses dan persetujuan pasien yang ketat. Kontrol akses merupakan komponen fundamental dalam perlindungan data pasien karena memastikan hanya pengguna berwenang yang dapat membaca, memodifikasi, atau membagikan informasi medis. Namun, penelitian sistematis menunjukkan bahwa sebagian besar sistem rekam medis elektronik (RME) masih menghadapi kendala dalam aspek otorisasi, akuntabilitas, dan akses darurat, serta minim dukungan terhadap mekanisme manajemen persetujuan pasien yang efektif \cite{cobrado2024access}. Model tradisional seperti \textit{Role-Based Access Control (RBAC)} dinilai tidak memadai untuk menangani konteks klinis yang dinamis dan memerlukan keputusan akses berdasarkan kondisi pasien, lokasi, serta urgensi waktu \cite{deoliveira2023acabac}.

Sebagai solusi, model \textit{Attribute-Based Access Control} (ABAC) dikembangkan untuk memungkinkan kontrol yang lebih spesifik dengan mempertimbangkan atribut pengguna, data, dan lingkungan. Studi oleh \cite{deoliveira2023acabac} memperkenalkan \textit{Acute Care Attribute-Based Access Control} (\textit{AC-ABAC}) yang menerapkan atribut kontekstual secara dinamis pada proses perawatan gawat darurat. Model ini memungkinkan sistem memberikan akses sementara kepada tim medis yang relevan tanpa mengorbankan privasi pasien, serta mencabut izin begitu sesi perawatan berakhir.

Namun, tantangan muncul pada praktik \textit{break-glass access}, yaitu mekanisme
pemberian akses darurat ketika nyawa pasien terancam. Pendekatan \textit{break-glass} tradisional yang hanya menonaktifkan kebijakan akses bersifat statis
terbukti berisiko disalahgunakan apabila tidak disertai mekanisme audit dan pencatatan forensik yang kuat \cite{deoliveira2023acabac}. Oleh karena itu, diperlukan
sistem yang mampu menyeimbangkan kebutuhan klinis dengan akuntabilitas
melalui penerapan kontrol akses dinamis, audit yang tidak dapat diubah, dan
notifikasi pasien.

Di sisi lain, kemunculan konsep \textit{Patient-Accessible Electronic Health Records}
(\textit{PAEHR}) dan \textit{patient portal} memperkuat paradigma perawatan yang berpusat pada
pasien, di mana pasien berperan aktif dalam mengontrol siapa yang dapat
mengakses data pribadinya dan untuk tujuan apa. Studi tinjauan cakupan oleh \cite{kariotis2025mentalhealth} menemukan bahwa akses pasien terhadap catatan medisnya
meningkatkan transparansi dan kepercayaan terhadap tenaga medis, sekaligus
mendorong komunikasi dua arah. Namun, hal ini juga memunculkan kekhawatiran
terhadap praktik dokumentasi dan perlindungan informasi sensitif dalam konteks
kesehatan mental.

Di Indonesia, penerapan SATUSEHAT masih mengandalkan persetujuan umum
dan kontrol akses yang bersifat umum dan kasar, sehingga pasien belum memiliki
mekanisme kendali granular atau spesifik terhadap akses data sensitif. Padahal,
regulasi nasional seperti Undang-Undang Nomor 27 Tahun 2022 tentang
Perlindungan Data Pribadi (PDP) dan Peraturan Menteri Kesehatan Nomor 24
Tahun 2022 tentang Rekam Medis mewajibkan penerapan prinsip keamanan,
kerahasiaan, keutuhan, serta hak pasien untuk menarik dan menghapus
persetujuan. Tanpa sistem yang mampu menegakkan kebijakan akses berbasis
konteks dan melacak aktivitas akses secara transparan, risiko pelanggaran privasi
dan sengketa hukum tetap tinggi meskipun platform nasional telah mengadopsi
standar interoperabilitas modern.

Berdasarkan permasalahan tersebut, penelitian ini akan merancang dan
mengevaluasi prototipe \textit{Consent \& Policy Gateway} yang menegakkan kebijakan
akses granular menggunakan \textit{FHIR} dan \textit{Data Segmentation for Privacy}
(\textit{DS4P}) \textit{Security Labels}. Sistem ini juga akan menyediakan \textit{portal} pasien
untuk mengatur pemberian atau pencabutan persetujuan, serta menerapkan
protokol \textit{break-glass} dengan audit yang tidak dapat diubah. Pendekatan ini
diharapkan dapat memenuhi tuntutan regulasi nasional sekaligus meningkatkan
transparansi dan kepercayaan pasien terhadap pengelolaan data rekam medis
elektronik di Indonesia.

% --- Rumusan Masalah ---
\section{Rumusan Masalah}
Berdasarkan latar belakang yang telah dijelaskan sebelumnya, berikut merupakan rumusan masalah tugas akhir ini:
\begin{enumerate}
    \item	Bagaimana merancang mekanisme persetujuan pasien yang bersifat
          granular dalam sistem rekam medis elektronik berbasis FHIR?
    \item	Bagaimana menegakkan kebijakan akses dan pelabelan keamanan untuk
          melindungi data sensitif pasien sesuai prinsip DS4P?
    \item	Bagaimana memastikan akses darurat \textit{(break-glass access)} dapat dilakukan
          secara aman, terkontrol, dan terdokumentasi secara forensik?
\end{enumerate}

% --- Tujuan ---
\section{Tujuan}
Berdasarkan masalah yang dirumuskan, berikut merupakan tujuan yang ingin
dicapai dalam pelaksanaan Tugas Akhir ini:
\begin{enumerate}
    \item Merancang dan mengimplementasikan prototipe \textit{Consent \& Policy Gateway}
          berbasis standar \textit{FHIR} untuk pengelolaan persetujuan \textit{granular} pasien.

    \item Mengintegrasikan dan menguji penerapan \textit{security labels} \textit{DS4P} guna
          menegakkan kebijakan akses data medis sensitif.

    \item Mengembangkan mekanisme \textit{break-glass} dan \textit{audit trail} yang tidak dapat
          diubah untuk menjamin akuntabilitas akses darurat.
\end{enumerate}

% --- Batasan Masalah ---
\section{Batasan Masalah}
Ruang lingkup dari permasalahan Tugas Akhir ini dibatasi agar tidak terjadi
penyimpangan bahasan penelitian dan memastikan tujuan tercapai. Berikut
merupakan batasan masalah pada pelaksanaan Tugas Akhir ini:
\begin{enumerate}
    \item Penelitian hanya berfokus pada validasi fungsionalitas perancangan dan
          implementasi prototipe (\textit{proof of concept}), bukan sistem produksi yang
          terintegrasi dengan SATUSEHAT atau sistem rumah sakit sebenarnya.

    \item Implementasi sistem difokuskan pada lapisan aplikasi website dan
          \textit{middleware} (\textit{Consent \& Policy Gateway}) tanpa mencakup
          pengembangan sistem rekam medis penuh dari sisi klinis.

    \item Simulasi dilakukan menggunakan dataset RME \textit{dummy} berbasis struktur
          \textit{FHIR Resources} (\textit{Patient}, \textit{Observation}, \textit{Consent},
          \textit{AuditEvent}, \textit{Provenance}), bukan data pasien nyata.

    \item Portal pasien dan portal klinisi dibangun dalam bentuk antarmuka web
          sederhana untuk demonstrasi konsep, sehingga desain antarmuka bersifat
          minimal dan fungsional, bukan fokus utama penelitian.

    \item Penelitian tidak mencakup implementasi kriptografi atau enkripsi data medis
          secara penuh dari awal, melainkan hanya berfokus pada kontrol akses dan
          pencatatan aktivitas.

    \item Evaluasi keamanan difokuskan pada konsistensi penegakkan kebijakan dan
          integritas audit, bukan pengujian penetrasi atau serangan siber.

    \item Pengembangan tidak termasuk aspek skalabilitas.
\end{enumerate}

% --- Metodologi Pengerjaan TA ---
\section{Metodologi}
Metodologi pelaksanaan Tugas Akhir ini menggunakan \textit{Software Development Life
    Cycle (SDLC)} dengan tahapan berikut:
\begin{enumerate}
    \item	Perencanaan\newline
          Tahap ini mencakup identifikasi kebutuhan sistem, penentuan ruang
          lingkup penelitian, serta penyusunan jadwal kerja. Aktivitas meliputi studi
          literatur terkait FHIR, DS4P, dan mekanisme kontrol akses pada rekam
          medis elektronik, serta penetapan alat, teknologi, dan batasan implementasi
          sesuai waktu pengerjaan.
    \item	Analisis\newline
          Pada tahap ini dilakukan analisis kebutuhan fungsional dan nonfungsional
          sistem, termasuk identifikasi aktor (pasien, klinisi, administrator), alur
          persetujuan, aturan kebijakan akses, dan skenario \textit{break-glass}. Analisis juga
          mencakup pemetaan atribut untuk penegakkan kebijakan berbasis FHIR dan
          simulasi DS4P \textit{security labels}.
    \item	Desain \newline
          Tahap desain berfokus pada perancangan arsitektur sistem, \textit{use case
              diagram} untuk menggambarkan interaksi pengguna dengan sistem, model
          basis data, serta desain modul utama seperti \textit{Consent Management}, \textit{Policy
              Engine}, \textit{Break-Glass Handler}, dan \textit{Audit Trail}. Selain itu, dibuat pula desain
          antarmuka portal pasien dan portal klinisi menggunakan prinsip kemudahan
          penggunaan serta pemetaan antar komponen \textit{backend} dan \textit{frontend}.
    \item	Implementasi\newline
          Implementasi dilakukan dengan mengembangkan prototipe \textit{Consent \&
              Policy Gateway} menggunakan tumpukan teknologi yang telah ditentukan.
          FHIR server diimplementasikan secara mock untuk mensimulasikan
          pertukaran data antar sistem, sementara \textit{DS4P security labels} diterapkan
          pada metadata sumber yang relevan.
    \item	Pengujian \newline
          Tahap ini bertujuan untuk memastikan fungsionalitas sistem berjalan sesuai
          kebutuhan melalui uji fungsional, uji kasus skenario akses, serta pengujian
          integritas jejak audit. Evaluasi dilakukan dengan menilai akurasi keputusan
          akses, keutuhan pencatatan audit, dan waktu respon sistem untuk
          memastikan prototipe berfungsi sesuai rancangan.
\end{enumerate}