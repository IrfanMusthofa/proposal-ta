% ============================================================================================
% BAB III ANALISIS MASALAH
% Pembagian subbab tidak rigid dan dapat bervariasi. Bab ini minimal berisi analisis kebutuhan
% fungsional dan nonfungsional, analisis berbagai alternatif solusi yang dapat ditawarkan, dan
% metode pemilihan solusi yang diusulkan.
% ============================================================================================
\chapter{ANALISIS MASALAH}
\label{chap:analisis-masalah}
\section{Analisis Kondisi Saat Ini}
Sistem rekam medis elektronik (RME) di Indonesia saat ini sedang bertransformasi
menuju interoperabilitas nasional melalui platform \textit{SATUSEHAT}, yang
mengadopsi standar \textit{FHIR (Fast Healthcare Interoperability Resources)}. Meskipun
langkah ini memperkuat pertukaran data antar fasilitas kesehatan, penerapan
mekanisme privasi dan kontrol akses yang memadai masih terbatas pada tingkat
persetujuan umum (\textit{general consent}). Artinya, pasien memberikan persetujuan
secara menyeluruh tanpa dapat menentukan data spesifik mana yang dapat diakses
oleh tenaga kesehatan tertentu (Ayaz dkk.\ 2021; Tabari dkk.\ 2024).

Selain itu, sistem \textit{SATUSEHAT} belum menerapkan segmentasi privasi berbasis
\textit{DS4P (Data Segmentation for Privacy)} untuk membedakan tingkat sensitivitas data
medis. Kondisi ini berpotensi menimbulkan pelanggaran privasi ketika data sensitif
seperti rekam kesehatan mental atau penyakit menular dibagikan secara luas tanpa
pembatasan yang proporsional.

Pada sisi keamanan, penerapan kontrol akses di fasilitas kesehatan umumnya masih
menggunakan model \textit{RBAC (Role-Based Access Control)} yang bersifat statis dan
hierarkis, sehingga sulit menyesuaikan dengan situasi dinamis seperti akses darurat
(\textit{emergency access}) atau kerja lintas departemen (de Carvalho Jr.\ \& Bandiera-Paiva
2018). Belum adanya jejak audit (\textit{audit trail}) yang tidak dapat diubah dan
kurangnya partisipasi pasien dalam pengelolaan persetujuan memperkuat perlunya
pendekatan baru berbasis manajemen persetujuan yang terperinci (\textit{fine-grained
    consent management}) dan penegakan kebijakan (\textit{policy enforcement}) dinamis.

Menurut dokumentasi resmi \cite{satusehat2025}, keamanan data pasien
dijaga melalui kebijakan hak akses yang memastikan hanya tenaga kesehatan di
fasilitas layanan yang memperoleh persetujuan pasien yang dapat mengakses data
tersebut. Selain itu, \textit{SATUSEHAT Platform} menerapkan metode pengamanan seperti
\textit{masking} dan \textit{encryption} untuk melindungi data selama pemrosesan dan
pertukaran. Hal ini menunjukkan bahwa meskipun mekanisme dasar perlindungan
data telah diterapkan, aspek pengelolaan persetujuan granular dan segmentasi privasi
masih belum didukung secara penuh.

\section{Analisis Kebutuhan}
\subsection{Identifikasi Masalah Pengguna}
Masalah utama yang dihadapi pengguna baik pasien maupun tenaga kesehatan dapat
diidentifikasi sebagai berikut:

\begin{enumerate}
    \item Kurangnya kontrol pasien atas data pribadi.\newline
          Pasien tidak dapat menentukan secara spesifik siapa yang boleh mengakses
          data tertentu sesuai kebutuhan medis.

    \item Ketergantungan pada persetujuan umum (\textit{general consent}).\newline
          Tidak ada mekanisme granular untuk mengelola izin akses berdasarkan tipe
          data, tujuan, atau waktu.

    \item Model kontrol akses yang kaku.\newline
          \textit{RBAC} tidak mendukung konteks darurat atau multi-atribut seperti lokasi
          dan kondisi klinis.

    \item Ketiadaan audit forensik yang transparan.\newline
          Aktivitas akses data belum dilengkapi jejak audit yang tidak dapat diubah
          (\textit{immutable audit trail}) untuk memastikan akuntabilitas.

    \item Risiko penyalahgunaan \textit{break-glass}.\newline
          Akses darurat dapat dilakukan tanpa pembatasan waktu atau otentikasi
          tambahan.

    \item Belum adanya portal pasien interaktif.\newline
          Sistem belum memberikan sarana bagi pasien untuk memberikan atau
          mencabut persetujuan secara langsung berbasis \textit{FHIR Consent}.
\end{enumerate}

Untuk mencari solusi atas masalah-masalah tersebut, perlu disusun kebutuhan
fungsional dan nonfungsional sistem yang diperlukan. Subbab berikut menjabarkan
kebutuhan-kebutuhan tersebut.

\clearpage

\subsection{Kebutuhan Fungsional}
Berikut adalah kebutuhan fungsional yang disajikan dalam bentuk tabel:
\begin{table}[H]
    \caption{Kebutuhan Fungsional}
    \label{tbl:kebutuhan-fungsional}
    \begin{tabular}{ | p{1.7cm} | p{3.2cm} | p{7.5cm} | }
        \hline
        \textbf{Kode} & \textbf{Kebutuhan Fungsional}  & \textbf{Deskripsi}                                               \\
        \hline

        FR-1          & Manajemen Persetujuan          &
        Sistem dapat mencatat, menampilkan, dan memperbarui status persetujuan pasien berbasis \textit{FHIR Consent}.     \\
        \hline

        FR-2          & Pemberian Persetujuan Granular &
        Pasien dapat menentukan akses berdasarkan jenis data, peran pengguna, dan tujuan penggunaan.                      \\
        \hline

        FR-3          & \textit{Policy Enforcement}    &
        Sistem menegakkan kebijakan akses secara otomatis menggunakan \textit{security labels DS4P} dan atribut pengguna. \\
        \hline

        FR-4          & \textit{Break-Glass Access}    &
        Tenaga medis dapat melakukan akses darurat dengan autentikasi tambahan dan batas waktu.                           \\
        \hline

        FR-5          & \textit{Audit Trail}           &
        Semua aktivitas akses dicatat secara kronologis dan dihash untuk memastikan integritas data audit.                \\
        \hline

        FR-6          & Portal Pasien \& Klinik        &
        Tersedia antarmuka web untuk pasien dan tenaga medis guna mengelola dan meninjau status akses.                    \\
        \hline
    \end{tabular}
\end{table}

\clearpage

\subsection{Kebutuhan Nonfungsional}
Berikut adalah kebutuhan nonfungsional yang disajikan dalam bentuk tabel:
\begin{table}[H]
    \caption{Kebutuhan Nonfungsional}
    \label{tbl:kebutuhan-nonfungsional}
    \begin{tabular}{ | p{1.7cm} | p{3.5cm} | p{7.3cm} | }
        \hline
        \textbf{Kode} & \textbf{Kebutuhan Nonfungsional} & \textbf{Deskripsi}                             \\
        \hline

        NFR-1         & Keamanan                         &
        Sistem menggunakan autentikasi dan hashing audit untuk menjaga kerahasiaan serta integritas data. \\
        \hline

        NFR-2         & Kinerja                          &
        Respon kebijakan akses tidak melebihi 10 detik pada skenario pengujian lokal.                     \\
        \hline

        NFR-3         & Auditabilitas                    &
        Semua keputusan akses dapat dilacak dengan identitas, waktu, dan alasan.                          \\
        \hline
    \end{tabular}
\end{table}

\section{Analisis Pemilihan Solusi}
\subsection{Alternatif Solusi}
Berikut adalah alternatif solusi yang dirancang dalam bentuk tabel:
\begin{table}[h]
    \caption{Alternatif Solusi}
    \label{tbl:alternatif-solusi}
    \begin{tabular}{ | p{1.5cm} | p{3.5cm} | p{7.5cm} | }
        \hline
        \textbf{Kode} & \textbf{Alternatif Solusi}                                           & \textbf{Deskripsi Singkat}                                                 \\
        \hline

        S-1           & \textit{RBAC Enhanced}                                               &
        Pengembangan sistem berbasis peran dengan tambahan lapisan verifikasi pasien, namun tetap bersifat statis dan kurang adaptif terhadap konteks.                    \\
        \hline

        S-2           & \textit{ABAC Policy Engine}                                          &
        Penerapan kontrol akses berbasis atribut dan integrasi \textit{FHIR Consent}, mendukung granularitas tinggi pada keputusan akses.                                 \\
        \hline

        S-3           & \textit{ABAC} dengan \textit{DS4P} dan Protokol \textit{Break-Glass} &
        Integrasi \textit{Attribute-Based Access Control} untuk kondisi darurat, ditambah pelabelan \textit{DS4P} serta mekanisme \textit{audit} yang tidak dapat diubah. \\
        \hline
    \end{tabular}
\end{table}
\newline
Ketiga alternatif ini dievaluasi untuk menentukan solusi yang paling sesuai dengan
kebutuhan keamanan, privasi, dan keterlibatan pasien.

\clearpage

\subsection{Analisis Penentuan Solusi}
Untuk menentukan solusi terbaik, ketiga alternatif dinilai berdasarkan beberapa
kriteria utama, yaitu: privasi data, transparansi \& auditabilitas, kemudahan
implementasi, kepatuhan regulasi, dukungan \textit{FHIR Consent}, dukungan akses
darurat, dan kesesuaian terhadap masalah penelitian.

Penilaian menggunakan skala 1–5, di mana:

1 = Sangat Buruk, 2 = Buruk, 3 = Cukup, 4 = Baik, 5 = Sangat Baik.

Berikut adalah hasil analisis penentuan solusi dalam bentuk tabel:

\begin{table}[H]
    \caption{Tabel III.3.1 Analisis Penentuan Solusi}
    \label{tbl:analisis-solusi}
    \begin{tabular}{ | p{4cm} | p{2cm} | p{2.2cm} | p{3cm} | }
        \hline
        \textbf{Kriteria Penilaian}                 &
        \textbf{S-1 \textit{RBAC Enhanced}}         &
        \textbf{S-2 \textit{ABAC Policy Engine}}    &
        \textbf{S-3 \textit{ABAC} + \textit{DS4P} + Protokol \textit{Break-Glass}} \\
        \hline

        Privasi Data                                &
        3                                           &
        4                                           &
        5                                                                          \\
        \hline

        Transparansi \& Auditabilitas               &
        2                                           &
        4                                           &
        5                                                                          \\
        \hline

        Kemudahan Implementasi                      &
        5                                           &
        4                                           &
        3                                                                          \\
        \hline

        Kepatuhan terhadap Regulasi
        (\textit{UU PDP 2022}, Permenkes 24/2022)   &
        3                                           &
        4                                           &
        5                                                                          \\
        \hline

        Dukungan \textit{FHIR Consent}              &
        2                                           &
        4                                           &
        5                                                                          \\
        \hline

        Dukungan Akses Darurat                      &
        1                                           &
        5                                           &
        3                                                                          \\
        \hline

        Kesesuaian terhadap Permasalahan Penelitian &
        3                                           &
        4                                           &
        5                                                                          \\
        \hline

        \textbf{Total Skor (dari 35)}               &
        \textbf{19}                                 &
        \textbf{27}                                 &
        \textbf{33}                                                                \\
        \hline
    \end{tabular}
\end{table}

\clearpage

Berdasarkan hasil evaluasi, solusi S-3 (\textit{ABAC} dengan \textit{DS4P} dan Protokol
\textit{Break-Glass}) memperoleh skor tertinggi (33/35) dan dinilai paling sesuai dengan
konteks penelitian. Solusi ini tidak hanya menegakkan privasi dan keamanan pasien
melalui penerapan \textit{security labels DS4P}, tetapi juga mendukung akses dinamis pada
situasi darurat dengan mekanisme \textit{audit} yang tidak dapat diubah. Selain itu, integrasi
\textit{FHIR Consent} memberi ruang bagi pasien untuk mengelola persetujuan secara
granular, memenuhi prinsip transparansi, akuntabilitas, dan kepatuhan regulasi
nasional.
